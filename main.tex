\documentclass{article}
\usepackage[utf8]{inputenc}
\usepackage[utf8]{inputenc}
\usepackage[russian]{babel}
\usepackage{setspace}
\usepackage{enumitem}
\usepackage[margin=0.5in]{geometry}
\setstretch{1.1}

\title{QUIZ_2}
\date{  }

\begin{document}

\maketitle

\section*{Ramsey model}

\begin{itemize}[noitemsep]
        \item Бесконечно живущие домохозяйства 
        \item S – норма сбережения, эндогенна 
        \itemДецентрализованное равновесие совпадает с социальным планировщиком 
        \item (K*;с*) -  стационарное состояние модели \textbf{Ramsey}
        \item C’t – получено из уравнения Эйлера 
        \itm K’t – из бюджетного ограничения 
        \item Формула для нормы сбережения 
        \item $S^*=\frac{\left(\eta+\delta\right)\alpha}{\rho+\eta+\delta}$
\end{itemize}


	1. В модели Рамсея равновесная норма сбережений тем выше, чем выше темп роста населения (да) 
	
	2. В модели Рамсея равновесная норма сбережений тем выше, чем ниже темп роста населения (нет) 

	Нет, так как при росте $\etha$ или $\delta$,   $S^*=\frac{\left(\eta+\ \delta\right)\alpha}{\rho+\eta+\delta}$. $\ethat$ и $\delta$ - являются восстановительными инвестициями, то есть из-за $\ethat$ и $\delta$ появляется необходимость делать сбережения, чтобы поддерживать K*(бюджетное органичение) $K = \frac{K}{L}$ => убытие  k из-за L$\uparrow$, a из-за $\delta$ - k$\downarrow$ (чем выше $\delta$, тем ниже K)

	3. В модели Рамсея равновесная норма сбережений тем ниже, чем выше норма cубъективных межвременных предпочтений (т.е. чем более домохозяйства нетерпеливы) (да) 
	
	4. В модели Рамсея равновесная норма сбережений тем выше, чем выше норма субъективных межвременных предпочтений (т.е. чем более домохозяйства нетерпеливы) (нет) 
	
	$S^*=\frac{\left(\eta+\ \delta\right)\alpha}{\rho+\eta+\delta}$, $\rho$ - норма субьективных межвременных предпочтений. 
	$\rho\uparrow => S^*\downarrow$, чем выше норма субьективных межвременных предпочтений ($\rho$), тем больше будет равновесная ставка (r) => стационарное потребление - constant. 
	
	Из уравнения Эйлера следует:
	$\frac{u'(C_{t}}{u'(C_{t+1})} = \frac{(1+r_{t+1})}{(1+r)(1+\rho)} =>$
	
	$\uparrow r =  f'(k) - \delta => k^*\downarrow => c^*\downarrow$ and $y^* \downarrow$, причем y* падает сильнее чем c*, => $S^*= [1-\frac{c*}{y*}]\downarrow$. Не из-за уравнения Эйлера, а из уравнения равновесной ставки !!! $\uparrow r =  f'(k) - \delta => k*\downarrow => c*\downarrow$!!!

	
	5. Т.к. домохозяйства нетерпеливы (дисконтируют полезность будущего потребления), равновесная норма сбережений в модели Рамсея \textbf{ниже}, чем норма сбережений, обеспечивающая соблюдение золотого правила накопления капитала в модели Солоу (да) 
	6. Т.к. домохозяйства нетерпеливы (дисконтируют полезность будущего потребления), равновесная норма сбережений в модели Рамсея \textbf{выше}, чем норма сбережений, обеспечивающая соблюдение золотого правила накопления капитала в модели Солоу (нет) 
	
	7. Т.к. описание процесса накопления капитала в моделях Солоу и Рамсея \textbf{совпадает}, равновесная норма сбережений в модели Рамсея совпадает с нормой сбережений, обеспечивающей соблюдение золотого правила накопления капитала в модели Солоу (нет) 
	
	$S_{GR} = \alpha \text$ из модели Solow, $S^*=\frac{\left(\eta+\delta\right)\alpha}{\rho+\eta+\delta}$  => $S_{GR}>S^*$, это связанно с тем, что $\rho$ влияет на равновесный уровень капитала $\uparrow r =  f'(k) - \delta => k*\downarrow => c*\downarrow$ and $y* \downarrow$, причем y* падает сильнее чем c*, => $S*= [1-\frac{c*}{y*}]\downarrow$.
	
	8. Стационарная норма сбережений в модели Рамсея отрицательно зависит от нормы субъективных межвременных предпочтений, что следует из уравнения Эйлера: чем выше  , тем меньше домохозяйства ценят будущее потребление и меньше сберегают (нет) 
	
	9. Стационарная норма сбережений в модели Рамсея положительно зависит от нормы субъективных межвременных предпочтений: чем выше , тем ниже стационарный уровень капиталовооруженности. В свою очередь, снижение капиталовооруженности приводит к большему снижению в потреблении, чем в выпуске на душу населения (нет) 
	
	$S^*=\frac{\left(\eta+\ \delta\right)\alpha}{\rho+\eta+\delta}$, $\rho$ - норма субьективных межвременных предпочтений. 
	$\rho\uparrow => S*\downarrow$, чем выше норма субьективных межвременных предпочтений ($\rho$), тем больше будет равновесная ставка (r) => стационарное потребление - constant. 
	
	Из уравнения Эйлера следует:
	$\frac{u'(C_{t}}{u'(C_{t+1})} = \frac{(1+r_{t+1})}{(1+r)(1+\rho)} =>$
	
	$\uparrow r =  f'(k) - \delta => k*\downarrow => c*\downarrow$ and $y* \downarrow$, причем y* падает сильнее чем c*, => $S*= [1-\frac{c*}{y*}]\downarrow$. Не из-за уравнения Эйлера, а из уравнения равновесной ставки !!! $\uparrow r =  f'(k) - \delta => k*\downarrow => c*\downarrow$!!!
	
	10. Стационарный уровень капиталовооруженности, соответствующий модифицированному золотому правилу накопления капитала в модели Рамсея, \textbf{ниже}, чем уровень капиталовооруженности, соответствующий простому золотому правилу в модели Солоу (да) 
	
	11. Стационарный уровень капиталовооруженности, соответствующий модифицированному золотому правилу накопления капитала в модели Рамсея, \textbf{выше}, чем уровень капиталовооруженности, соответствующий простому золотому правилу в модели Солоу (нет) 
	
	12. Стационарная ставка процента, определяемая модифицированным золотым правилом накопления капитала в модели Рамсея, \textbf{ниже} стационарной ставки процента, соответствующей простому золотому правилу в модели Солоу (нет) 
	
	13. Стационарная ставка процента, определяемая модифицированным золотым правилом накопления капитала в модели Рамсея, \textbf{выше} стационарной ставки процента, соответствующей простому золотому правилу в модели Солоу (да)
	

#graph
	
	\begin{itemize}[noitemsep]
    \item Модифицированное GR(MGR) - является обычным равновесием в модели Ramsey
    \item $f'(k^*) = \rho+\eta+\delta$, уравнение локуса (на графике)
    \item $f'(k^{GR}) = \eta + \delta $
	\end{itemize}

	
	В модели Solow S - экзогенная переменная, в модели Ramsey S - эндогенная переменная с учетом $\rho$.  
	
	=> $f'(k^*) > f'(k^{GR}), => k^* < k^{GR}, => r* > r^{GR}$ 
	
	14. Равновесие в децентрализованной экономике в модели Рамсея является Парето-эффективным (да) 
	
	15. Равновесие в децентрализованной экономике в модели Рамсея не соответствует максимально возможному уровню стационарного потребления на душу населения, и, следовательно, не является Парето-эффективным (нет) 
	
	16. Равновесие в децентрализованной экономике в модели Рамсея необязательно совпадает с решением задачи поиска командного оптимума (нет)
	
	\begin{enumerate}[noitemsep]
	\item В модели Ramsey Парето-оптимум там, где максимизируется полезность единственного домохозяйства.
	\item Социальный планировщик имеет такое же бюджетное органичение(BC), что и домохозяйство.
	\item Социальный планировщик имеет такую же функцию полезности, что и домохозяйство
	\end{enumerate}
	
	BC(decentralized): $C_t + (1+\eta)k_{t+1} - k_t = w_t - r_t k_t$, где $w_t - r_t k_t = f (k_t) - \delta k_t$
	
	BC(Social planner): $(1+n)k_{t+1} = f(k_t) - C_t + k_t(1-\delta)$ =>
	
	=> Максимальное возможное C - это не Парето-оптимум, так как не учитывается нетерпимость Домохозяйств.
	
	Общественное благосостояние против децентрализованное равновесие. 
	\begin{enumerate}
	\item Бюджетное ограничение ДХ  $C_t+(K_{t+1} - K) = w_t L_t + r_t K_t$
	\item Разделим на $L_t$ и перепишем в терминах на душу населения. 
		$c_t+(1+n)k_{t+1} - k = w_t+r_t k_t$
	\item В равновесии на рынках факторов цены факторов равны их предельным продуктам $w_t = f(k_t) - c_t + k_t(1-\delta)$
	\item В итоге приходим к ресурсному ограничению экономики 
	$(1+n)k_{t+1} = f(k_t) - c_t + k_t(1-\delta)$
	\end{enumerate}
	
    Динамическое бюджетное ограничение ДХ сводится к ресурсному ограничению экономики, с котрорым сталкивается социальный планировщик. Следовательно, децентрализованное равновесие в экономике(выбор потребления и накопления капитала) является Парето-эффективным, потому что выбор социального планировщика (Командный оптимум) является Парето-эффектиным по постоению. Это означает, что любое вмешательство фискальных властей не может приводить к Парето-улучшению. 
	
	17. Фискальная политика в модели Рамсея \textbf{приводит к отклонению} от Парето-эффективного равновесия (да) 
	
	18. В то время как аккордные налоги в модели Рамсея \textbf{не приводят к Парето-ухудшению}, искажающие налоги – приводят (нет) 

    Изначальное равновесие в модели Ramsey Парето-оптимально(доказать) => любое вмешательство в экономику государством ухудшает ее состояние, так как исключает оптимум (являющийся по определению наилучшим решением системы) который домохозяйства получили исходя из собственных предпочтений.  
	
	
	19. В модели Рамсея государственные закупки, финансируемые за счет аккордных налогов, приводят к полному вытеснению \textbf{потребления} в стационарном состоянии (да) 
	
	20. В модели Рамсея государственные закупки, финансируемые за счет аккордных налогов, приводят к полному вытеснению \textbf{инвестиций} в стационарном состоянии (нет) 
	
	Локус $C't$ не сдвигается, так как ставка r не искажается => уравнение Эйлера. 
	
	Локус $K't$ сдвигается вниз, так как сбор налогов скоращает бюджетное ограничение домохозяйств. 
	\begin{itemize}[noitemsep]
	    \item Государственные закупки вытесняют потребление, так как происходит сдвиг $K't$ (на величину $\Delta G$)
	    \item Государственные закупки не вытесняют инвестиции, так как не изменится $K^*$ (потому что не происходит сдвиг $C't$), уровень $K^*$ поддерживается.  $\Delta G = C_0^{R}-C_{1}$
	\end{itemize}

\section*{OLG Model (Overlapping generations)}

\begin{itemize}[noitemsep]
    \item Домохозяйство живет \textbf{2} периода. 
    \item В каждый момент времени живет и молодое и старое Домохозяйство.
    \item $C_1 t$ - потребление молодого ДХ, $C_2 {t+1}$ - потребление того же ДХ, но в периоде "2" (старое).
\end{itemize}
\begin{enumerate}
    \item $\frac{U'(C_1 t)}{U'(C_2 t)} = 1 +\eta$ 
    \item $\frac{U'(C_i t)}{U'(C_i t+1)} = \frac{1 + r_{t+1}}{(1+\eta)(1+\rho)}$
    \item $\frac{U'(C_1 t)}{U'(C_2 t+1)} = \frac{1+r_{t+1}}{1+\rho}$
\end{enumerate}
В командном оптимуме модели OLG выполняются равенства \textbf{1, 2, 3}. В децентрализованном только \textbf{3} равенство. Децентрализованное домохозяйство отличается от командного оптимума.

1. Равновесие в децентрализованной экономике перекрывающихся поколений характеризуется тем же модифицированным золотым правилом накопления капитала, что и равновесие в модели Рамсея (нет)

Равновесие в децентрализованной экономике - $\frac{U'(C_1 t)}{U'(C_2 t+1)} = \frac{1+r_{t+1}}{1+\rho}$, в стационарном состоянии $C_1 ^* , C_2 ^2 , r^* $, где $C_1 ^* = C_1 (t=1) + ... + C_1 (t=i) + ... C_1 (t=n)$ =>

$C_2 ^* = C_2 (t=1) + ... + C_2 (t=i) + ... C_2 (t=n)$ =>

 => $\frac{U'(C_1 ^*)}{U'(C_2 ^*)} = \frac{1+r_{t+1} ^*}{1+\rho}$ => соответствует MGR(Modified Golden Rule) только  если $\frac{U'(C_1 ^*)}{U'(C_2 ^*)} = 1+\eta$ и  $1+\eta = frac{1+r^2}{1+\rho}$, => $r^* \approx \eta + \rho$.

В Ramsey: $f'(K_{MGR}(k^*)) = \eta + \delta + \rho$, => $f'(K_{MGR}(k^*))-\delta = \eta + \rho$, где $r_{MGR} = f'(K_{MGR}(k^*))-\delta$. В остальных случаях равновесия модели OLG не будет соответствовать MRG.

2. Командный оптимум в экономике перекрывающихся поколений характеризуется тем же модифицированным золотым правилом накопления капитала, что и равновесие в модели Рамсея (да)

$C_1 ^* = C_1 (t=1) + ... + C_1 (t=i) + ... C_1 (t=n)$

$C_2 ^* = C_2 (t=1) + ... + C_2 (t=i) + ... C_2 (t=n)$

В Ramsey: $f'(K_{MGR}(k^*)) = \eta + \delta + \rho$, => $f'(K_{MGR}(k^*))-\delta = \eta + \rho$, где $r_{MGR} = f'(K_{MGR}(k^*))-\delta$. В Командном оптимуме (социальный планировщик) $\frac{U'(C_i t)}{U'(C_i t+1)} = \frac{1 + r_{t+1}}{(1+\eta)(1+\rho)}$ в стационарном состоянии $C_1 ^* , C_2 ^2 , r^* $, => $U'(C_i t)= U'(C_i t+1)$ => $1 = \frac{1 + r_{t+1} ^*}{(1+\eta)(1+\rho)}$ отсюла следует, что $r^* \approx \eta + \rho$. В Ramsey $f'(K_{MGR}(k^*)) = \eta + \delta + \rho$ => $r_{MGR}= \eta + \rho$, так как $r = f'(k) - \delta$. Следовательно КО = MGR

3. В задаче поиска командного оптимума в модели перекрывающихся поколений, выбор оптимального соотношения между потреблением сосуществующих в одном периоде молодого и пожилого поколений зависит от соотношения численности поколений (да)

Да, так как $1+\eta$ - темп роста поколений; соотношение численности поколений то есть $C_1 t= (1+\eta)C_2 t$

4. В задаче поиска командного оптимума в модели перекрывающихся поколений, выбор оптимального соотношения между потреблением молодых поколений в двух последовательных периодах зависит только от соотношения между ставкой процента и нормой субъективного дисконтирования и не зависит от соотношения численности поколений (нет)


5. В задаче поиска командного оптимума в модели перекрывающихся поколений, выбор оптимального соотношения между потреблением пожилых поколений в двух последовательных периодах зависит только от соотношения между ставкой процента и нормой субъективного дисконтирования и не зависит от соотношения численности поколений (нет)

Метода как во втором вопросе! $\frac{U'(C_i t)}{U'(C_i t+1)} = \frac{1 + r_{t+1}}{(1+\eta)(1+\rho)}$ + дописать 

6. В задаче поиска командного оптимума в модели перекрывающихся поколений, выбор оптимального соотношения между потреблением отдельного поколения в двух соседних периодах времени определяется тем же правилом Рамсея-Кейнса, что и выбор в задаче самого домохозяйства. Это означает, что и оптимальные уровни потребления в двух задачах совпадают (нет)

Нет, так как $\frac{U'(C_1 t)}{U'(C_2 t+1)} = \frac{1+r_{t+1}}{1+\rho}$ выполняется как в Командном оптимуме так и в децентрализованной экономике, однако в Командном оптимуме также выполняются $\frac{U'(C_1 t)}{U'(C_2 t)} = 1 +\eta$  и $\frac{U'(C_i t)}{U'(C_i t+1)} = \frac{1 + r_{t+1}}{(1+\eta)(1+\rho)}$.   В Децентрализованной экономике Домохозяйства рассматривают доходы $(w, r)$ - как экзогенно заданные и следовательно не задаются вопросом о том, как накопленный капитал влияет на будущие поколения. 

В Командном оптимуме доходы $(w, r)$ являются эндогенными переменными и учитывают накопленный капитал, то есть социальный планировщик смотрит на домохозяйства в совокупности. 

7. Природа динамической неэффективности в децентрализованной экономике перекрывающихся поколений заключается в установлении неэффективно высокого уровня потребления на душу населения и недостаточно высокого накопления капитала (нет)

8. Природа динамической неэффективности в децентрализованной экономике перекрывающихся поколений заключается в перенакоплении капитала: существует возможность увеличить потребление на душу населения в одном или нескольких периодах за счет сокращения запаса капитала (да)

Природа Динамической неэффективности в перенакоплении капитала, то есть запас капитала может быть излишние большим и его сокращение может увеличить потребление в одном или сразу нескольких периодах, такое может произойти если $r^* < \eta$. $c^* = f(k^*) - (\eta + \delta)k^*$, таким образом $\frac{d e ^*}{d k^*} = f'(k^*) - (\eta + \delta)$, Динамическая неэффективоность происходит, когда $\frac{dc^*}{dk^*} < 0$. $f'(k^*) - (\eta + \delta) => f'(k^*)-\delta < \eta $, где $r^* = f'(k^*)-\delta $ => $r^* < \eta$.

\textbf{Теория по пенсионной сиситеме:}
\begin{itemize}
    \item \textbf{Накопительная} \textbf{ПС}. Государство забирает $m_t$ и кладет в актив с доходностью $r_{t+1}$. В следующем периоде государство возвращает $m+(1+r_{t+1}$.
    \item \textbf{Распределительная ПС}. $m_t$ забирается у молодых и отдается старым в следующем периоде $m_t(1+n)$ 
\end{itemize}

9. Если обязательные отчисления в накопительную пенсионную систему не превышают величину добровольных сбережений, которые домохозяйства делали бы в отсутствии пенсионной системы, то такая пенсионная система не отражается на равновесии в экономике перекрывающихся поколений (да)

$S_t$ - Добровольные вложения.
$M_t$ - Обязательные вложения.
Накопительная ПС - обязательные вложения в активы с такой же нормой доходности, как у добровольных вложений. При этом $(S_t + M_t)$ не меняются, то есть доюровольные вложения вытесняют обязательные 1:1. То есть не оказывается никакого влияния на сумму сбережений =>  Обязательные отчисления не оказывают влияния на накопление капитала и равновесие.

10. Если отчисления в распределительную пенсионную систему не превышают величину добровольных сбережений, которые домохозяйства делали бы в отсутствии пенсионной системы, то такая пенсионная система не отражается на равновесии в экономике перекрывающихся поколений (нет)

Отчисления в распределительной ПС - принудительные вложения в актив с доходностью $r$. Принудительные отчисления $M_t$ вытесняют добровольные отчисления $S_t$, но не в пропорции 1:1.

$\eta > r => S\downarrow$ сильное падение $S$

$\eta < r => S\downarrow$ слабое падение $S$

Более того, в каждый момент времени часть средств не переходит в инвестиции, а просто перекладывается из одного кармана в другой ( от молодых к старым ) => Равновесный капитал \downarrow.

11. И накопительная, и распределительная пенсионные системы приводят к сокращению добровольных сбережений домохозяйств (да)

Да так как сужаеются Бюджетные ограничения.

12. Отчисления из зарплаты в рамках распределительной пенсионной системы сокращают добровольные сбережения домохозяйств больше чем один к одному (нет)

13. Распределительная пенсионная система означает перераспределение доходов между разными домохозяйствами, но это не означает снижение совокупных сбережений в экономике (нет)

14. Распределительная пенсионная система замедляет накопление капитала в экономике и дает меньшую стационарную капиталовооруженность, что однозначно означает Парето- ухудшение по сравнению со случаем государственного невмешательства (нет)

 В каждый момент времени часть средств не переходит в инвестиции, а просто перекладывается из одного кармана в другой ( от молодых к старым ) => Равновесный капитал $\downarrow$. Но не обязательно Парето-ухудшение, так как может быть обратное при условии динамической неэффективности капитала. 

15. Замедление роста населения снижает отдачу внутри распределительной пенсионной системы (да)

Да, так как на пожилое поколение будет приходиться большая норма средств при быстром росте населения.

\section*{LCH Model}

Рассмотрим репрезентативное домохозяйство, рожденное в период $t$, его жизнь имеет продолжительность $T$, которая в свою очередь разделена на два периода: рабочий и пенсионный ($T = L_{Labor} + R_{Rest}$).

Трудовой доход является экзогенно заданной величиной, положителен и постоянен на протяжении $L$ рабочих лет и равен нулю на пенсии. 

\begin{itemize}[noitemsep]
 
\item $y_{t+\tau} ^t = y^t, 0 <= \tau <= L-1$, 

\item $y_{t+\tau} ^t= 0, L<=\tau <= T-1$

\item $T_R<=\tau<= T-1$
\end{itemize}

Индекс $t$ означает рождение, $\tau$ - возраст, $t+\tau$ - текущий период

\begin{itemize}[]
    \item ДХ максимизирует функцию интегральной полезности $\sum_{\tau = 0} ^{T-1}\frac{u(c_{t+\tau} ^t)}{(1+\rho)^\tau} \rightarrow max_{c_{t+\tau} ^t}$ 
    \item Для простоты предположим, что ставка процента и норма субьективных межвременных предпочтений равны нулю, тогда межвременное бюджетное ограничение имеет вид: 
    
    $\sum_{\tau = 0} ^{T-1} c_{t+\tau} ^t = \sum_{\tau = 0} ^{T-1} y_{t+\tau} ^t$
    \item Уравнение Эйлера дает нам постоянную траекторию потребления:  $\frac{u'(c_{t+\tau} ^t)}{u'(c_{t+\tau+1} ^t)} = \frac{1+r}{1=\rho}, r = \rho = 0, => c_{t+\tau} ^t = c_{t+\tau+1} ^t = c^t,  \forall \tau$


\end{itemize}
, 

1. Теория жизненного цикла предполагает, что доходы экономических агентов на протяжении всей их жизни возрастают (нет) 

2. В теории жизненного цикла рациональное поведение индивидуумов состоит в сглаживании траектории потребления (да) 

3. В теории жизненного цикла временной профиль потребительских расходов имеет форму идентичную профилю заработанного дохода (нет) 

4. В теории жизненного цикла временной профиль потребительских расходов имеет меньший «горб», чем профиль заработанного дохода (да) 

5. Статистические исследования показывают, что профили частных (добровольных) и пенсионных (принудительных) сбережений в жизненном цикле качественно схожи (нет) 

6. Статистические исследования показывают, что частные (добровольные) сбережения в жизненном цикле являются положительной величиной для любого возраста (да) 

7. Статистические исследования показывают, что пенсионные (принудительные) сбережения в жизненном цикле являются положительной величиной для любого возраста (нет) 

8. Статистические исследования показывают, что общие (частные и пенсионные) сбережения в жизненном цикле являются положительной величиной для любого возраста (нет) 

9. Статистические исследования, подтверждая теорию жизненного цикла, показывают, что общее богатство имеет «горбообразный» возрастной профиль (да) 

10. Статистические исследования показывают, что в отличие от частного богатства, пенсионное богатство не имеет «горбообразный» возрастной профиль (нет) 

11. В теории жизненного цикла пенсионное богатство экономических агентов может рассматриваться как аннуитет (да) 

\begin{itemize}
        \item После выхода на пенсию ДХ проедает свое богатство: при $y_{t+\tau} ^t= 0, L<=\tau <= T-1$
        
        $c^t = \frac{L}{L+R}y^t$, при $0 <= \tau <= L-1$

        \itemДХ накапливает активы в первые $L$ лет жизни. 
        $S_{t+\tau} ^t = \frac{R}{R+L}y^t$, при $0 <= \tau <= L-1$

        $a_{t+\tau} ^t =  \tau S_{t+\tau} ^t =  \frac{R}{R+L}y^t\tau$, при  $0 <= \tau     <= L$

        \item Их величина достигает своего максимума к моменту выхода на пенсию: $a_{t+L} ^t =  \tau_{t+\tau} ^t =  \frac{LR}{R+L}y^t$
        
        \item Поделив свое накопленное богатство на $R$ лет пенсии, ДХ имеет возможность поддерживать постоянный уровень потребления и дальше.
        $s_{t+\tau} ^t= -\frac{a_{t+\tau} ^t}{R} = -\frac{L}{L+R}y^t = \frac{L(R+L+\tau)}{L+R}y^t$, при $L <= \tau <= T-1$
        
\end{itemize}

12. В теории жизненного цикла рациональные экономические агенты, решая задачу межвременной оптимизации потребления, должны рассматривать только частные (добровольные) сбережения, т.к. пенсионные (принудительные) сбережения формируют богатство как аннуитет (нет) 

 ДХ максимизирует функцию интегральной полезности $\sum_{\tau = 0} ^{T-1}\frac{u(c_{t+\tau} ^t)}{(1+\rho)^\tau} \rightarrow max_{c_{t+\tau} ^t}$.  
Для простоты предположим, что ставка процента и норма субьективных межвременных предпочтений равны нулю, тогда межвременное бюджетное ограничение имеет вид: 
    $\sum_{\tau = 0} ^{T-1} c_{t+\tau} ^t = \sum_{\tau = 0} ^{T-1} y_{t+\tau} ^t$. 
 
 Уравнение Эйлера дает нам постоянную траекторию потребления:  $\frac{u'(c_{t+\tau} ^t)}{u'(c_{t+\tau+1} ^t)} = \frac{1+r}{1=\rho}, r = \rho = 0, => c_{t+\tau} ^t = c_{t+\tau+1} ^t = c^t,  \forall \tau$

13. Теория жизненного цикла дает микроэкономическое обоснование кейнсианской функции потребления (да) 

Агрегированное потребление $C(t) = \frac{L}{L+R}Y(t) + \int_{t-L}^{t-L-R}\frac{L}{L+R}y(\tau)\etha(\tau)d\tau$, где $ \frac{L}{L+R}Y(t)$ - потребление рабочих домохозяйств, а $\int_{t-L}^{t-L-R}\frac{L}{L+R}y(\tau)\etha(\tau)d\tau$ - потребление домохозяйств на пенсии. Следовательно мы посторили микрообоснованную функцию потребления Кейнса, где агрегированное потребление - линейная функция от трудового дохода, а автономное потребление - проедание накопленного богатства. 

14. В теории жизненного цикла норма агрегированного потребления отличается от индивидуальной на коэффициент, зависящий от темпа экономического роста (да) 

15. В теории жизненного цикла норма агрегированного потребления совпадает с нормой индивидуального потребления (нет) 

Норма агрегированного потребления - \textbf{apc} = $\frac{C(t)}{Y(t)} = \frac{L}{L+R}\frac{1-e^{1(g_A + n)(L+R)}}{1-e^{-(g_A + n)L}}$, где $g_A + n = g(y)$ - темп прироста экономики, а $\frac{L}{L+R}$ - индивидуальная норма, $g_A$ - темп прироста производительности труда, $n$ - темп прироста населения. => агрегированная норма потребления apc отличается от индивидуальной, так как зависит от темпов прироста экономики. 

16. В теории жизненного цикла для определения нормы агрегированных сбережений, не имеет значение, происходит ли более быстрый рост населения или более быстрый рост производительности – значение имеет только их сумма (да) 

Да так как $aps = 1 - apc = 1 - \frac{C(t)}{Y(t)} = 1- \frac{L}{L+R}\frac{1-e^{1(g_A + n)(L+R)}}{1-e^{-(g_A + n)L}}$, отсюда $(g_A + n) = g(y)$, следовательно на \textbf{apc} влияет сумма факторов, которая выражется в приросте экономики. 

17. В теории жизненного цикла норма агрегированных сбережений является возрастающей вогнутой сверху функцией от темпа экономического роста (да) 

18. В теории жизненного цикла норма агрегированных сбережений является убывающей вогнутой сверху функцией от темпа экономического роста (нет) 

В теории жизненного цикла норма агрегированных сбережений является возрастающей вогнутой сверху функцией от темпа экономического роста, потому что $aps = 1 - apc = 1 - \frac{C(t)}{Y(t)} = 1- \frac{L}{L+R}\frac{1-e^{1(g_A + n)(L+R)}}{1-e^{-(g_A + n)L}}$
 = > при росте темпов экономического прироста $g(y)$ растет и норма агрегированных сбережений. $g(y)=0 aps = 0 $, $g(y)=0.01 aps = 0.045 $, $g(y)=0.03 aps = 0.111 $, $g(y)=0.05 aps =0.151  $. 

19. Более высокие темпы роста экономики приводят к большей норме агрегированных сбережений при неизменной индивидуальной норме сбережений работающего репрезентативного агента (да) 

20. Более высокие темпы роста экономики приводят к меньшей норме агрегированных сбережений при неизменной индивидуальной норме сбережений работающего репрезентативного агента (нет) 

\textbf{Агрегированная APS}

\begin{itemize}[noitemsep]
    \item Чем быстрее растет экономика, тем выше агрегированная норма сбережений при неизменной индивидуальной норме сбережений.
    \item Если экономика совсем не расет $g_Y = 0$, то агрегированная норма сбережений равна нулю! (Внимание на казуальнсть! $g_Y = 0 => aps = 0$, а не наоборот).
    \item Агрегированные сбережения положительны, если поток положительных сбережений работающих ДХ превышает отрицательные сбережения пенсионеров. Следует из роста населения (Neisser effect) и роста трудового дохода (Bentzel effect).
\end{itemize}

21. В отсутствии экономического роста модель жизненного цикла предсказывает отсутствие 
агрегированных сбережений: индивидуальные положительные сбережения работающих поколений в точности компенсируются отрицательными сбережениями пенсионных поколений (да) 

22. Положительные агрегированные сбережения существуют только тогда, когда положительные сбережения работающих поколений превышают (по абсолютной величине) отрицательные сбережения неработающих поколений (да) 

23. Положительные агрегированные сбережения существуют только тогда, когда и молодые, и старые поколения делают положительные сбережения (нет) 

Определим агрегированные сбережения как: 

$S(t) = Y(t) - C(t)  = \int_{t-L}^t\frac{R}{L+R}y_0 e^{g_{A^{\tau}}} n_0 e^{n\tau}d\tau +[- \int_{t-L-R}^{t-L}\frac{R}{L+R}y_0 e^{g_{A^{\tau}}} n_0 e^{n\tau}d\tau]$, 

где $\int_{t-L}^t\frac{R}{L+R}y_0 e^{g_{A^{\tau}}} n_0 e^{n\tau}d\tau$ - Positive savings of working households, 

a $[-\int_{t-L-R}^{t-L}\frac{R}{L+R}y_0 e^{g_{A^{\tau}}} n_0 e^{n\tau}d\tau]$ - Negative savings of retirees. 

24. Положительные агрегированные сбережения могут рассматриваться как результат роста численности поколений и роста производительности труда (да) 

$aps = 1 - apc = 1 - \frac{C(t)}{Y(t)} = 1- \frac{L}{L+R}\frac{1-e^{1(g_A + n)(L+R)}}{1-e^{-(g_A + n)L}}$, Агрегированная средняя склонность к сбережению - возрастающая функция от темпа роста экономики.

25. Из гипотезы жизненного цикла следует, что не норма агрегированных сбережений определяет темп роста выпуска, а наоборот (да) 

Да, так как основной резуальтат модели LCH заключается в определении направления причинно-следственной связи между экономическим ростом и нормой сбережений. Экономический рост определяет норму сбережений, а не наоборот! 

26(1). Из гипотезы жизненного цикла следует, что норма агрегированных сбережений определяет темп роста выпуска 

27(2). Подтверждая предсказания теории жизненного цикла, статистические исследования показывают наличие положительной взаимосвязи между темпами роста экономики и нормой сбережений 

28(3). Современные эмпирические исследования показывают наличие причинно-следственной связи, идущей от высоких темпов экономического роста к высоким уровням сбережений, и не показывают аналогичной обратной зависимости 


29(4). Современные эмпирические исследования показывают наличие причинно-следственной связи, идущей от высоких уровней сбережений к высоким темпам экономического роста, и не показывают аналогичной обратной зависимости 

30(5). Альтернативное объяснение положительного воздействия темпа роста на норму сбережений на микро-уровне опирается на гипотезу формирования привычек в потреблении: домохозяйства не сразу реагируют ростом потребления на увеличение доходов. Тем самым рост доходов может стимулировать рост сбережений (да)

В теории жизненного цикла норма агрегированных сбережений является возрастающей вогнутой сверху функцией от темпа экономического роста, потому что $aps = 1 - apc = 1 - \frac{C(t)}{Y(t)} = 1- \frac{L}{L+R}\frac{1-e^{1(g_A + n)(L+R)}}{1-e^{-(g_A + n)L}}$
 = > при росте темпов экономического прироста $g(y)$ растет и норма агрегированных сбережений. $g(y)=0 aps = 0 $, $g(y)=0.01 aps = 0.045 $, $g(y)=0.03 aps = 0.111 $, $g(y)=0.05 aps =0.151  $.
 
 
 \textbf{Модель PIH - Permanent Income Hypothesis }
 \begin{itemize}[noitemsep]
     \item 1. $A_0 + Y_0^D + \frac{Y_1^D}{1+r} + \frac{Y_2^D}{(1+r)^2} + .... + \frac{Y_n^D}{(1+r)^n} = Y^{PIH} + \frac{ Y^{PIH}}{1+r} + \frac{ Y^{PIH}}{(1+r)^2}  + .... + \frac{ Y^{PIH}}{(1+r)^n} = \frac{1+r}{r}Y^{PIH}$, где $Y^{PIH}$ - приведенная стоимость доходов. 
     \item 2. $A_0 + \sum \frac{Y^D}{(1+r)^n}$
     \item 3. $Y^{PIH} = \frac{r}{1+r} [A_0 + \sum \frac{Y^D}{(1+r)^n}]$
     \item $Y^{PIH}$ - Приведенная стоимость доходов будующих периодов приведенная в расчете на один период. Аннулететная стоимость богатства.  
 \end{itemize}
 
31(6). Перманентный доход определяется как приведенная стоимость всех будущих доходов (нет)

32(7). Перманентный доход определяется как часть приведенной стоимости всех будущих доходов, приходящуюся на один период жизни (да)

33(8). Перманентный доход определятся как аннуитетная величина (платеж) для общего богатства, которым обладает домохозяйство

34(9). Перманентный доход – это постоянная величина, определяемая как среднеарифметический 
доход домохозяйства

31(6). - 34(10).  смотри -> Модель PIH - Permanent Income Hypothesis 


35(10).В случае совпадения (постоянных во времени) значений ставки процента и нормы субъективных 
межвременных предпочтений, теория перманентного дохода М. Фридмана предсказывает 
постоянный уровень потребительских расходов (да)

36(11).В случае совпадения (постоянных во времени) значений ставки процента и нормы субъективных 
межвременных предпочтений, теория перманентного дохода М. Фридмана предсказывает 
независимость уровня потребительских расходов от ставки процента (нет)

В случае совпадения (постоянных во времени) значений ставки процента и нормы субъективных 
межвременных предпочтений $r = \rho$ => $\frac{U'(C_1 t)}{U'(C_2 t+1)} = \frac{1+r}{1+\rho} = 1, => C_t = C_{t+1}$

$C_0 + \frac{C_1}{1+r} + \frac{C_2}{(1+r)^2}  + .... + \frac{ Y^{C_n}(1+r)^n}$, => $c^* \frac{1+r}{r} = A_0 + \sum\frac{Y_D}{(1+r)^n}, => c^*  = \frac{r}{1+r}[A_0 + \sum\frac{Y_D}{(1+r)^n}]$, 

где $c^* \approx Y^{PIH}$

37(12).В модели перманентного дохода временный доход определяется как разность между текущим и перманентным доходом, т.е. сбережения со знаком минус (нет)

$Y_t^T =  Y_t ^D - Y^{PIH} = S_t$, где $Y^{PIH} = c_t$

38(13). Теория М. Фридмана предсказывает, что уровень потребления определяется величиной временный дохода (нет) 

35(10). - 36(11).  $\uparrow\uparrow\uparrow$

39(14). Если потребительские расходы определяются величиной перманентного дохода, временное изменение в налогообложение будет оказывать меньшее воздействие на потребление, по сравнению с перманентным изменением (да) 

40(15). Если потребительские расходы определяются величиной перманентного дохода, временное изменение в налогообложение будет оказывать большее воздействие на потребление, по сравнению с перманентным изменением (нет) 

Влияние изменения налогообложения оказывает меньшее влияниена приведенную стоимость доходов, чем на перманентное изменение в доходах, следовательно меньше влияние на $Y^{PIH}$.

41(16). В отличие от теории жизненного цикла, гипотеза перманентного дохода утверждает, что экономические агенты сберегают или заимствуют, чтобы сгладить временную траекторию потребительских расходов (нет) 



42(17). Теория жизненного цикла, как и гипотеза перманентного дохода, утверждает, что экономические агенты сберегают или заимствуют, чтобы сгладить временную траекторию потребительских расходов (да) 

В обеих теориях доходы не постоянны во времени, сбережения выступают в роли инструмента перераспределения доходов для сглаживания потребления во времени.

43(18). В соответствии с гипотезой перманентного дохода изменения во временной структуре располагаемого дохода, не воздействующие на величину перманентного дохода, не отражаются на динамике сбережений домохозяйств (нет) 

44(19). В соответствии с гипотезой перманентного дохода изменения во временной структуре располагаемого дохода, не воздействующие на величину перманентного дохода, не отражаются на динамике потребления домохозяйств (да) 

\begin{enumerate}
    \item $C_i = a + bY_i + e_i$
    \item $\hat{b} = \frac{Cov(Y,C)}{Var(Y)} = \frac{Cov(Y^P + Y^T, YT^P}{Var(Y^P + Y^T)} = \frac{Var(Y^P)}{Var(Y^P)+Var(Y^T)}$
    \item $\hat{a} = \bar{C} - \hat{b}\bar{Y} = \bar{Y^P} - \hat{b}(\bar{Y^P} + \bar{Y^T}) = (1-\hat{b})\bar{Y^P}$
\end{enumerate}

Ecли $Var(Y^T) >> Var(Y^P)$, то изменения в текущем доходе в большей степени отражают изменения во временном доходе. То есть, эффект на потребление невелик (i. e. $1 < a, 0 < a$). 

Ecли $Var(Y^T) << Var(Y^P)$, то изменения в текущем доходе в большей степени отражают изменения в перманентном доходе. То есть, потребление меняется в пропорции 1:1 с текущим доходом. 






\end{document}
